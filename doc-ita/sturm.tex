\documentclass[twoside,10pt]{article}

\makeatletter
\def\input@path{{:macros:}{macros/}{:chapters:}{chapters/}}
\makeatother

\usepackage[a4paper]{geometry}
\usepackage{color,times,bbm}
\usepackage[italian]{babel}

\usepackage[math]{easyeqn}
\usepackage[definevectors,spacesep]{easyvector}
%:

\definecolor{headercolor}{rgb}{0.07,0.41,0.28}
\definecolor{sectioncolor}{rgb}{0.25,0.1,0.5}
\definecolor{subsectioncolor}{rgb}{0.25,0.25,0.75}
\definecolor{bodycolor}{rgb}{0,0,0}

\usepackage[thmmarks]{ntheorem}
\theoremnumbering{arabic}
\theoremstyle{plain}
%begin{latexonly}
\theoremheaderfont{\normalfont\sffamily\bfseries\color{headercolor}}
\theoremseparator{.}
%end{latexonly}
\newtheorem{proposizione}{Proposizione}
\newtheorem{teorema}{Teorema}
\newtheorem{lemma}[teorema]{Lemma}
\newtheorem{corollario}[teorema]{Corollario}
%
%begin{latexonly}
\theorembodyfont{\normalfont\sffamily\color{bodycolor}}
\theoremheaderfont{\normalfont\sffamily\bfseries\color{headercolor}}
\theoremseparator{.}
%end{latexonly}
\newtheorem{definizione}{Definizione}
\newtheorem{osservazione}{Osservazione}
\newtheorem{esempio}{Esempio}
\newtheorem{problema}{Problema}
\newtheorem{applicazione}{Applicazione}
\newtheorem{esercizio}{Esercizio}
% 
\theoremstyle{nonumberplain}
%begin{latexonly}
\theoremsymbol{\ensuremath{\rule{2mm}{2mm}}}
\theorembodyfont{\small\normalfont\sffamily}
\theoremheaderfont{\normalfont\sffamily\bfseries}
\theoremseparator{.}
\theorembodyfont{\normalfont\sffamily\color{bodycolor}}
\theoremheaderfont{\normalfont\sffamily\bfseries\color{headercolor}}
%end{latexonly}
\newtheorem{dimostrazione}{Dimostrazione}
\newtheorem{soluzione}{Risoluzione}

\DeclareMathAlphabet\mathcal{U}{eus}{m}{n}
\DeclareMathAlphabet\mathbcal{U}{eus}{b}{n}

\let\bbb\mathbbm
\let\Cal\mathcal
\let\BCal\mathbcal

\let\ds\displaystyle
\let\IMP\Rightarrow
\let\IMPD\Downarrow
\let\SSE\Longleftrightarrow

%begin{latexonly}
\usepackage{sectsty}
\sectionfont{\nohang\normalfont\Large\sffamily\bfseries\color{sectioncolor}}
\subsectionfont{\nohang\normalfont\large\sffamily\bfseries\color{subsectioncolor}}
%end{latexonly}

\usepackage{appunti}

\renewcommand{\baselinestretch}{1.2}
\parindent0pt\relax
\parskip0.4\baselineskip plus 2pt\relax

\title{Zeri di funzioni e teorema di Sturm}
\author{Enrico Bertolazzi}
\date{\normalfont\itshape
     Dipartimento di Ingegneria Meccanica e Strutturale\\
     Universit\`a degli Studi di Trento\\
     via Mesiano 77, I -- 38050 Trento, Italia\\
     \texttt{Enrico.Bertolazzi@ing.unitn.it}
}

\begin{document}
    
\pagestyle{myfancy}
\def\chaptermark#1{\markboth{#1}{}}
\def\sectionmark#1{\markright{#1}}

\maketitle

%\section{Zeri di polinomi}

\section{Eliminazione delle radici multiple}
Il metodo di Newton-Raphson \`e un metodo generalmente del secondo
ordine, ma se vogliamo approssimare una radice multipla di un
polinomio la convergenza degrada al primo ordine.  Un modo per evitare
questo degrado delle prestazioni \`e quello di sostituire al polinomio
che contiene radici multiple un polinomio con le stesse radici ma
semplici.  Questo problema apparentemente complicato ha una soluzione
molto semplice:
Sia $p(\xx[])$ un polinomio monico fattorizzato come segue%
\footnote{nella formula $\PROD[i]{j=1}{k}$ significa che vengono fatti 
tutti i prodotti con indice $j$ che va da $1$ a $k$ escluso $i$}
%%
\begin{EQ}
     p(\xx[]) = \prod_{i=1}^{k}(\xx[]-\xx[i])^{\mm[i]},
\end{EQ}
%%
allora il polinomio derivata prima si scriver\`a come
%%
\begin{EQ}[rcl]
    p'(\xx[]) &=& q(\xx[])\prod_{i=1}^{k}(\xx[]-\xx[i])^{\mm[i]-1}, \\
    q(\xx[])  &=& \sum_{i=1}^{k} \mm[i] \PROD[i]{j=1}{k}(\xx[]-\xx[i]),
\end{EQ}
%%
osserviamo che il polinomio $q(\xx[])$ non ha radici in comune con il 
polinomio $p(\xx[])$ infatti
%%
\begin{EQ}
   q(\xx[\ell]) = \sum_{i=1}^{k} \mm[i] \PROD[i]{j=1}{k}(\xx[\ell]-\xx[i])
   = \mm[\ell] \PROD[\ell]{j=1}{k}(\xx[\ell]-\xx[i]) \neq 0, \qquad
   \ell = 1,2,\ldots,k
\end{EQ}
%%
Quindi il polinomio $p(\xx[])$ e $p'(\xx[])$ hanno in comune solo le
radici di $p(\xx[])$ con molteplicit\`a maggiore di $1$.  Se $\xx[i]$
\`e una radice comune con molteplicit\`a $\mm[i]$ allora la sua
molteplicit\`a in $p'(\xx[])$ sar\`a $\mm[i]-1$.

Ricordiamo che un polinomio $M(\xx[])$ \`e il massimo comun divisore
tra due polinomi $P(\xx[])$ e $Q(\xx[])$ se vale
%%
\begin{enumerate}
    \item $M(\xx[])$ divide $P(\xx[])$ e $Q(\xx[])$, cio\`e
    %%
    \begin{EQ}\label{eq:newton:020}
        P(\xx[]) = M(\xx[]) S(\xx[]), \qquad
        Q(\xx[]) = M(\xx[]) T(\xx[]),
    \end{EQ}
    %%
    dove $S(\xx[])$ e $T(\xx[])$ sono a loro volta polinomi (anche 
    di grado $0$)
    
    \item Se $N(\xx[])$ \`e un altro polinomio che divide $P(\xx[])$ e
    $Q(\xx[])$ allora $N(\xx[])$ divide $M(\xx[])$ cio\`e
    %%
    \begin{EQ}
        M(\xx[]) = W(\xx[]) N(\xx[]),
    \end{EQ}
    %%
    per $W(\xx[])$ opportuno polinomio.
\end{enumerate}

\begin{teorema}
    Se $M(\xx[])$ \`e il massimo comun divisore tra due polinomi
    $P(\xx[])$ e $Q(\xx[])$ e se $\alpha$ \`e una radice comune a
    $P(\xx[])$ e $Q(\xx[])$ cio\`e:
    %%
    \begin{EQ}
        P(\alpha)= 0, \qquad Q(\alpha) =0,
    \end{EQ}
    %%
    allora necessariamente $\alpha$ \`e anche una radice di $M(\xx[])$.
\end{teorema}
\begin{dimostrazione}
    Se cos{\`\i} non fosse dalla \eqref{eq:newton:020} avremo che il
    polinomio $\xx-\alpha$ divide sia $S(\xx[])$ che $T(\xx[])$ cio\`e
    %%
    \begin{EQ}
        P(\xx[]) = M(\xx[]) (\xx-\alpha) S^{1}(\xx[]), \qquad
        Q(\xx[]) = M(\xx[]) (\xx-\alpha) T^{1}(\xx[]),
    \end{EQ}
    %%
    e $M(\xx[]) (\xx-\alpha)$ sarebbe un divisore comune a $S(\xx[])$
    e $T(\xx[])$ che non divide $M(\xx[])$.
\end{dimostrazione}

\begin{teorema}
    Se $\alpha$ \`e una radice di $M(\xx[])$ massimo comun divisore
    tra i polinomi $P(\xx[])$ e $Q(\xx[])$ allora $\alpha$ \`e una
    radice comune a $P(\xx[])$ e $Q(\xx[])$ cio\`e:
    %%
    \begin{EQ}
        P(\alpha)= 0, \qquad Q(\alpha) =0,
    \end{EQ}
    %%
\end{teorema}
\begin{dimostrazione}
    dalla \eqref{eq:newton:020} segue immediatamente
    %%
    \begin{EQ}[rcl]
        P(\alpha) &=& M(\alpha) S(\alpha) = 0, \\
        Q(\alpha) &=& M(\alpha) T(\alpha) = 0,
    \end{EQ}
    %%
\end{dimostrazione}
%%
Conseguenza di questi due teoremi \`e che il massimo comun divisore
tra due polinomi \`e costituito dal prodotto delle radici in comune
con molteplicit\`a il minimo tra le due.
%%
\begin{esempio}
    Siano
    %%
    \begin{EQ}[rcl]
        P(\xx[]) &=& 3(\xx[]-1)^{3}(\xx[]+1)^{2}(\xx[]-3), \\
        Q(\xx[]) &=& 2(\xx[]-1)^{2}(\xx[]+1)(\xx[]-3)^{3},
    \end{EQ}
    %%
    il loro massimo comun divisore \`e
    %%
    \begin{EQ}
        M(\xx[]) = (\xx[]-1)^{2}(\xx[]+1)(\xx[]-3).
    \end{EQ}
\end{esempio}

Sia $m(\xx[])$ il massimo comun divisore tra i polinomi $p(\xx[])$ e
$p'(\xx[])$ allora per i discorsi precedenti vale
%%
\begin{EQ}
     m(\xx[]) = \prod_{i=1}^{k}(\xx[]-\xx[i])^{\mm[i]-1}.
\end{EQ}
%%
Allora il polinomio
%%
\begin{EQ}
    \dfrac{p(\xx[])}{m(\xx[])} = \prod_{i=1}^{k}(\xx[]-\xx[i]),
\end{EQ}
%%
conterr\`a tutte le radici di $p(\xx[])$ e solo radici semplici.  Il
problema \`e come trovare questo massimo comun divisore.  Per fare
questo ci viene in aiuto un algoritmo classico, l'algoritmo di Euclide
(Euclide di Alessandria circa 365--300 a.C.) per il calcolo del
massimo comun divisore.
%%
\begin{algorithm}{Algoritmo di Euclide per il M.C.D. di $p(x)$ e $q(x)$}
{\label{algo:euclide}\qinput{$p(x)$, $q(x)$}}
  \qif {$\partial p>\partial q$} \\
  \qthen $p_{0}\qlet p$; $p_{1}\qlet q$ \\
  \qelse $p_{0}\qlet q$; $p_{1}\qlet p$
  \qfi \\
  $i \qlet 1$ \\
  \qrepeat \\
    \qcom{$p_{i+1}$ \`e il resto della divisione di $p_{i-1}$ con $p_{i}$.} \\
    \qcom{$p_{i-1}(x) = s_{i}(x) p_{i}(x) + p_{i+1}(x)$} \\
    $i \qlet i+1$
  \quntil {$p_{i}\equiv 0$} \\
  \qcom{$p_{i-1}$ \`e il massimo comun divisore.}
\end{algorithm}

\begin{esempio}
    Sia dato
    %%
    \begin{EQ}[rcl]
        p(x)  &=& (x-1)^{3}(x-2)(x+1)^{2}, \\
              &=&  x^{6} - 3 x^{5} + 6 x^{3} - 3 x^{2} - 3x+2, \\
        p'(x) &=&  6 x^{5} - 15 x^{4} + 18 x^{2} - 6 x  - 3,
    \end{EQ}
    %%
    calcoliamo il massimo comun divisore $m(x)$ con l'algoritmo di
    Euclide:
    \begin{enumerate}
        \item Inizializzazione:
        \begin{EQ}[rcl]
            p_{0}(x) &=& x^{6} - 3 x^{5} + 6 x^{3} - 3 x^{2} - 3x+2, \\
            p_{1}(x) &=& 6 x^{5} - 15 x^{4} + 18 x^{2} - 6 x  - 3,
        \end{EQ}
        \item Calcolo $p_{2}$:
        \begin{EQ}[rcl]
            p_{0}(x) &=& p_{1}(x) s_{1}(x) + p_{2}(x),\\
            x^{6} - 3 x^{5} + 6 x^{3} - 3 x^{2} - 3x+2 &=&
            \left(6 x^{5} - 15 x^{4} + 18 x^{2} - 6 x  - 3\right)
            \left(\dfrac{x}{6}-\dfrac{1}{12}\right)
            \\ && 
            + \left(
            \dfrac{7}{4} -  3x - \dfrac{x^{2}}{2} + 3 x^{3}
            - \dfrac{5}{4} x^{4}
            \right).
        \end{EQ}
        \item Calcolo $p_{3}$:
        \begin{EQ}[rcl]
            p_{1}(x) &=& p_{2}(x) s_{1}(x) + p_{3}(x), \\
            6 x^{5} - 15 x^{4} + 18 x^{2} - 6 x  - 3 &=&
            \left(-\dfrac{24 x}{5}+\dfrac{12}{25}\right)
            \left(
            \dfrac{7}{4} -  3x - \dfrac{x^{2}}{2} + 3 x^{3}
            - \dfrac{5}{4} x^{4}
            \right) \\
            && + \dfrac{96}{25}\left( -x^{3} + x^{2} + x -1 \right).
        \end{EQ}
        \item Calcolo $p_{4}$:
        \begin{EQ}[rcl]
            p_{2}(x) &=& p_{3}(x) s_{1}(x) + p_{4}(x), \\
            6 x^{5} - 15 x^{4} + 18 x^{2} - 6 x  - 3 &=&
            \left(\dfrac{125 x}{384}-\dfrac{175}{384}\right)
            \dfrac{96}{25}\left( -x^{3} + x^{2} + x -1 \right) + 0.
        \end{EQ}
        \item Poich\'e $p_{4}\equiv 0$ il massimo comun divisore \`e $p_{3}$
        \begin{EQ}
            \dfrac{96}{25}\left( -x^{3} + x^{2} + x -1 \right),
        \end{EQ}
        e poich\'e \`e unico a meno di una costante moltiplicativa
        scegliamo
        \begin{EQ}
            m(x) =-x^{3} + x^{2} + x -1.
        \end{EQ}
    \end{enumerate}
    %%
    dividendo $p(x)$ per $m(x)$ otteniamo
    %%
    \begin{EQ}
        \dfrac{p(x)}{m(x)} = -\dfrac{x^{3}}{2} + x^{2} + \dfrac{x}{2} -1.
    \end{EQ}
    %%
\end{esempio}
%%
Per questioni computazionali (ad esempio nella costruzione di
successioni di Sturm) a volte \`e pi\`u conveniente usare una versione
modificata dell'algoritmo tenendo cento del fatto che se $m(x)$ \`e il
massimo comun divisore tra due polinomi anche $c m(x)$ lo \`e dove $c$
\`e un qualunque scalare non nullo.  La versione \`e la seguente
%%
\begin{algorithm}{Algoritmo di Euclide per il M.C.D. di $p(x)$ e
$q(x)$ (versione modificata)}{ \qinput{$p(x)$, $q(x)$} } \qif
{$\partial p>\partial q$} \\
  \qthen $p_{0} \qlet p$; $p_{1} \qlet q$ \\
  \qelse $p_{0} \qlet q$; $p_{1} \qlet p$
  \qfi \\
  $i \qlet 1$ \\
  \qrepeat \\
     \qcom{ $p_{i+1}$ \`e il resto della divisione di $p_{i-1}$ 
            con $p_{i}$.}\\
     \qcom{ $p_{i-1}(x) = s_{i}(x) p_{i}(x) - c_{i} p_{i+1}(x)$}\\
    $i \qlet i+1$
  \quntil {$p_{i}\equiv 0$} \\
  \qcom{ $p_{i-1}$ \`e il massimo comun divisore}
\end{algorithm}
%%
Osserviamo che l'algoritmo~\ref{algo:euclide} produce effettivamente
il massimo comun divisore.  Vale infatti il seguente teorema:
%%
\begin{teorema}
    L'algoritmo~\ref{algo:euclide} termina in un numero finito $m$ di
    passi e l'ultimo resto non nullo \`e il massimo comun divisore dei
    polinomi.
\end{teorema}
\begin{dimostrazione}
    Innanzitutto osserviamo che l'algoritmo termina in un numero
    finito di passi.  Infatti il resto della divisione ha grado
    strettamente minore del divisore e quindi ad ogni divisione si
    riduce il grado dei polinomi coinvolti.  Quando il grado \`e $0$
    il polinomio \`e uno scalare.  La divisione di un polinomio per
    uno scalare ha resto nullo e quindi anche in questo caso estremo
    l' algoritmo termina.  L'algoritmo pu\`o essere quindi scritto
    come segue
    %%
    \begin{EQA}[rcl]
        \pp[0](x) &=& \ss[1](x)\pp[1](x) - c_{1}\pp[2](x),
        \label(~.0){eq:newton:eu:a} \\
        \pp[1](x) &=& \ss[2](x)\pp[2](x) - c_{2}\pp[3](x) ,
        \label(~.1){eq:newton:eu:b}\\
        &\vdots & \\
        \pp[k-1](x) &=& \ss[k](x)\pp[k](x) - c_{k}\pp[k+1](x),
        \label(~.k){eq:newton:eu:k} \\
        &\vdots&  \\
        \pp[m-2](x) &=& \ss[m-1](x)\pp[m-1](x) - c_{m-1}\pp[m](x) ,
        \label(~.m-1){eq:newton:eu:mu}\\
        \pp[m-1](x) &=& \ss[m](x)\pp[m](x), \label(~.m){eq:newton:eu:m}
    \end{EQA}
    %%
    e segue subito che $\pp[m](x)$ l'ultimo resto non nullo divide
    $\pp[m-1](x)$, $\pp[m-2](x)$,\ldots, $\pp[0](x)$ e quindi \`e un
    divisore comune.  Viceversa sia $q(x)$ un divisore di $\pp[0](x)$
    e $\pp[1](x)$, allora dalla \eqref{eq:newton:eu:a} segue se $q(x)$
    divide $\pp[2](x)$ e dalla \eqref{eq:newton:eu:b} segue che $q(x)$
    divide $\pp[3](x)$ e cos\`{\i} via fino ad arrivare alla
    \eqref{eq:newton:eu:m} da cui segue che $q(x)$ divide $\pp[m](x)$
    e quindi $\pp[m](x)$ \`e il massimo comun divisore.
\end{dimostrazione}


\subsection{Separazione delle radici}
E` desiderabile avere il modo per conoscere il numero di \emph{radici
reali} in un dato intervallo.  Se questo \`e possibile \`e possibile
tramite un metodo di bisezione separare tutte le radici reali e
approssimarle fino alla approssimazione voluta.  Questo problema pu\`o
essere risolto grazie alle successioni che prendono il nome dal suo
scopritore Jacques Charles Francois Sturm 1803--1855:
%%
\begin{definizione}\label{def:sturm}\index{Sturm!successione di}
    La successione di funzioni continue definite su $[a,b]$:
    %%
    \begin{EQ}
        \Cal{F}(x) = \{ \ff[0](x), \ff[1](x),\ldots,\ff[m](x) \},
    \end{EQ}
    %%
    forma una \emph{successione di Sturm su $[a,b]$} se  vale:
    %%
    \begin{enumerate}
        \item $\ff[0](x)$ ha al pi\`u zero semplici in $[a,b]$;
        \item $\ff[m](x)$ non ha zeri su $[a,b]$;
        \item per ogni $\alpha\in[a,b]$ tale che $\ff[k](\alpha)=0$, allora
        $\ff[k-1](\alpha)\ff[k+1](\alpha)<0$;
        \item per ogni $\alpha\in[a,b]$ tale che $\ff[0](\alpha)=0$, allora
        $\ff[0]'(\alpha)\ff[1](\alpha) < 0$;
    \end{enumerate}
\end{definizione}

\begin{definizione}
    Data una successione di Sturm $\Cal{F}(x)
    =\{\ff[0](x),\ff[1](x),\ldots,\ff[m](x)\}$ per ogni punti $x$
    associamo un vettore di $m+1$ numeri reali.  A questo vettore
    possiamo associare un numero intero detto variazione di segno. 
    Questo numero rappresenta il numero di volte che scorrendo la
    successione di numeri c'\`e un cambio di segno.  Ad esempio la
    successione
    %%
    \begin{EQ}
        \{ \underbrace{1, 0, -2}_{*}, \underbrace{-3, 4}_{*}, 3, 0, 1 \},
    \end{EQ}
    %%
    ha $2$ variazioni di segno (marcate con $*$). Osserviamo che $3,0,1$ 
    ha variazione di segno nulla. Infatti lo $0$ va considerato come 
    elemento neutro e va rimosso dal computo.
\end{definizione}
%%
Per ogni successione di Sturm vale il seguente teorema:
%%
\begin{teorema}[Sturm]\index{Sturm!teorema di}
    Data una successione di Sturm $\Cal{F}(x)
    =\{\ff[0](x),\ff[1](x),\ldots,\ff[m](x)\}$ su $[a,b]$ il numero di
    zeri di $\ff[0](x)$ in $(a,b)$ \`e dato dalla differenza tra il
    numero di variazioni di segno in $\Cal{F}(b)$ e $\Cal{F}(a)$.
\end{teorema}
\begin{dimostrazione}
    Il numero di variazioni di segno cambia man mano che $x$ passa da
    $a$ a $b$ solo a causa del cambio di segno di qualche funzione
    della successione di Sturm.  Assumiamo che per un qualche
    $\hat{x}\in(a,b)$ valga $\ff[k](\hat{x})=0$ per $0<k<m$.  In un
    intorno di $\hat{x}$ dalla condizione $3$ sar\`a
    %%
    \begin{center}
        \begin{tabular}{c|ccc}
            $x$ & $\ff[k-1](x)$ & $\ff[k](x)$ & $\ff[k+1](x)$ \\
            \hline
            $\hat{x}+\epsilon$ & $+$ & $\pm$ & $-$ \\
            $\hat{x}         $ & $+$ & $0$   & $-$ \\
            $\hat{x}-\epsilon$ & $+$ & $\pm$ & $-$ \\
            \hline
        \end{tabular}
        \qquad oppure \qquad 
        \begin{tabular}{c|ccc}
            $x$ & $\ff[k-1](x)$ & $\ff[k](x)$ & $\ff[k+1](x)$ \\
            \hline
            $\hat{x}+\epsilon$ & $-$ & $\pm$ & $+$ \\
            $\hat{x}         $ & $-$ & $0$   & $+$ \\
            $\hat{x}-\epsilon$ & $-$ & $\pm$ & $+$ \\
            \hline
        \end{tabular}
    \end{center}
    %%
    in ogni caso il passaggio da $\hat{x}$ non cambia il numero di
    variazioni di segno.  Sia ora $\hat{x}$ uno zero di $\ff[0](x)$ e
    vediamo la variazione di segno nell'intorno di $\hat{x}$,
    osserviamo che dalla condizione $4$ della
    definizione~\ref{def:sturm} avremo
    %%
    \begin{center}
        \begin{tabular}{c|cc}
            $x$ & $\ff[0](x)$ & $\ff[1](x)$ \\
            \hline
            $\hat{x}+\epsilon$ & $+$ & $+$\\
            $\hat{x}         $ & $0$ & $+$\\
            $\hat{x}-\epsilon$ & $-$ & $+$\\
            \hline
        \end{tabular}
        \qquad oppure \qquad 
        \begin{tabular}{c|cc}
            $x$ & $\ff[0](x)$ & $\ff[1](x)$ \\
            \hline
            $\hat{x}+\epsilon$ & $-$ & $-$ \\
            $\hat{x}         $ & $0$ & $-$ \\
            $\hat{x}-\epsilon$ & $+$ & $-$ \\
            \hline
        \end{tabular}
    \end{center}
    %%
    in ogni caso in numero delle variazioni di segno cresce al passare
    di $x$ per uno zero di $\ff[0](x)$.  Combinando queste
    osservazioni otteniamo il teorema.
\end{dimostrazione}


\subsubsection{Costruzione della successione di Sturm per un polinomio}
E` relativamente facile costruire una successione di Sturm per un
polinomio.  Sia $\ff[0]\equiv\pp[n]$ un polinomio di grado $n$,
definiamo $\ff[1]\equiv -\pp[n]'$.  Dividiamo $\ff[0](x)$ per
$\ff[1](x)$ e chiamiamo il resto $-\ff[2](x)$.  Poi dividiamo
$\ff[1](x)$ per $\ff[2](x)$ e chiamiamo il resto $-\ff[3](x)$. 
Continuiamo cos\`\i\ finch\'e il procedimento termina.  Abbiamo
cos\`\i\ la successione:
%%
\begin{EQ}[rcl]\label{eq:newton:021}
   \ff[0](x) &=& \qq[1](x)\ff[1](x) - \ff[2](x),\\
   \ff[1](x) &=& \qq[2](x)\ff[2](x) - \ff[3](x), \\
             &\vdots & \\
   \ff[k-1](x) &=& \qq[k](x)\ff[k](x) - \ff[k+1](x), \\
             &\vdots&  \\
   \ff[m-2](x) &=& \qq[m-1](x)\ff[m-1](x) - \ff[m](x), \\
   \ff[m-1](x) &=& \qq[m](x)\ff[m](x).
\end{EQ}
%%
Questa successione a parte il segno di $\ff[i]$ \`e esattamente la
successione per il calcolo del massimo comun divisore di un polinomio,
e $\ff[m]$ \`e il massimo comun divisore tra $\ff[0]$ e $\ff[1]$.  La
successione di polinomi
%%
\begin{EQ}
     \pp[i](x) = \dfrac{\ff[i](x)}{\ff[m](x)}, \qquad i=0,1,\ldots,m
\end{EQ} 
%%
\`e una successione di Sturm, infatti
%%
\begin{enumerate}
   \item $\pp[0](x)$ ha al pi\`u zero semplici in $[a,b]$;
         essendo il rapporto tra il polinomio originario ed il massimo 
         comun divisore tra il polinomio originario e la sua derivata 
         prima.
   \item $\pp[m](x)$ non ha zeri su $[a,b]$; infatti \`e una costante.
   \item per ogni $\alpha\in[a,b]$ tale che $\pp[k](\alpha)=0$, allora
         $\pp[k-1](\alpha)\pp[k+1](\alpha)<0$; infatti dalla \eqref{eq:newton:021}
         abbiamo
         \begin{EQ}
             \pp[k-1](\alpha) = -\pp[k+1](\alpha).
         \end{EQ}
         Osserviamo che se $\pp[k-1](\alpha) = \pp[k](\alpha) = \pp[k+1](\alpha)=0$ 
         allora dalla \eqref{eq:newton:021} seguirebbe 
   \item per ogni $\alpha\in[a,b]$ tale che $\pp[0](\alpha)=0$, allora
         $\pp[0]'(\alpha)\pp[1](\alpha)=-\pp[0]'(\alpha)^{2}<0$;
\end{enumerate}



\subsection{Limitazione delle radici}
Per poter usare il metodo di bisezione e separare le radici con Sturm
\`e necessario conoscere una prima stima anche se grossolana dell'
intervallo in cui stanno tutte le radici di un polinomio.  Per fare
questo occorre osservare che la seguente matrice in forma di Frobenius
(Ferdinand Georg Frobenius 1849--1917)
%%
\begin{EQ}
    \AA=\MAT{
      0 & 1 & 0 & 0 & \cdots & 0 \\
      0 & 0 & 1 & 0 & \cdots & 0 \\
      0 & 0 & 0 & 1 & & \vdots\\
      \vdots & & & \ddots & \ddots & 0\\
      0 &  0 & 0 & \cdots & 0 & 1 \\
      \dfrac{\aa[0]}{\aa[n]} & -\dfrac{\aa[1]}{\aa[n]} & \dfrac{\aa[2]}{\aa[n]}& \cdots & 
      (-1)^{n-1}\dfrac{\aa[n-2]}{\aa[n]} & (-1)^{n}\dfrac{\aa[n-1]}{\aa[n]}
    },
\end{EQ}
%%
ha come polinomio caratteristico
%%
\begin{EQ}\label{eq:newton:poly}
     \pp[](\lambda)=\DET{\AA-\lambda\II} = 
      \dfrac{(-1)^{n}}{\aa[n]}\left(
      \aa[0]+\aa[1]\Blambda[]+\aa[2]\Blambda[;2]+
       \cdots+\aa[n-2]\Blambda[;n-2]+\aa[n-1]\Blambda[;n-1]+\aa[n]\Blambda[;n]\right),
\end{EQ}
%%
(basta sviluppare lungo l'~ultima riga).  Applicando il teorema di
Gershgorin alla matrice otteniamo che le radici del polinomio
\eqref{eq:newton:poly} soddisfano
%%
\begin{EQ}
    \abs{\lambda-\dfrac{\aa[n-1]}{\aa[n]}}\leq \abs{\dfrac{\aa[0]}{\aa[n]}}+\abs{\aa[1]} + \cdots + 
    \abs{\dfrac{\aa[n-2]}{\aa[n]}},\qquad \abs{\lambda}\leq 1.
\end{EQ}
%%
Applicando il teorema di Gershgorin alla matrice trasposta otteniamo
che le radici del polinomio \eqref{eq:newton:poly} soddisfano
%%
\begin{EQ}
    \abs{\lambda}\leq \abs{\dfrac{\aa[0]}{\aa[n]}},\qquad
    \abs{\lambda}\leq \abs{\dfrac{\aa[i]}{\aa[n]}}+1,\qquad
    \abs{\lambda-\dfrac{\aa[n-1]}{\aa[n]}}\leq 1.
\end{EQ}
%%
Possiamo applicare le disegualianze appena viste (indebolendole un
poco) ad un polinomio qualunque,
%%
\begin{EQ}
    \pp[](\xx[]) = 
    \aa[0]+\aa[1]\xx[]+\aa[2]\xx[;2]+\cdots+\aa[n-1]\xx[;n-1]+\aa[n]\xx[;n],
\end{EQ} 
%%
ottenendo le seguenti limitazioni
%%
\begin{EQ}[rcl]
    \abs{\xx[]} &\leq& \max\left\{
         1, \sum_{i=0}^{n-1} \abs{\dfrac{\aa[i]}{\aa[n]}}
    \right\},
    \\
    \abs{\xx[]} &\leq& \max\left\{
         \abs{\dfrac{\aa[0]}{\aa[n]}},1+\max_{i=1}^{n-1}\abs{\dfrac{\aa[i]}{\aa[n]}}
    \right\}.
\end{EQ}
% 



\end{document}
